\documentclass[10pt,twocolumn,letterpaper]{article}

\usepackage{cvpr}
\usepackage{times}
\usepackage{epsfig}
\usepackage{graphicx}
\usepackage{amsmath}
\usepackage{amssymb}
\usepackage{biblatex}
\addbibresource{egbib.bib}


% Include other packages here, before hyperref.

% If you comment hyperref and then uncomment it, you should delete
% egpaper.aux before re-running latex.  (Or just hit 'q' on the first latex
% run, let it finish, and you should be clear).
\usepackage[breaklinks=true,bookmarks=false]{hyperref}

\cvprfinalcopy % *** Uncomment this line for the final submission

\def\cvprPaperID{****} % *** Enter the CVPR Paper ID here
\def\httilde{\mbox{\tt\raisebox{-.5ex}{\symbol{126}}}}

% Pages are numbered in submission mode, and unnumbered in camera-ready
%\ifcvprfinal\pagestyle{empty}\fi
\setcounter{page}{1}
\begin{document}

%%%%%%%%% TITLE
\title{Machine learning project: Speech recognition challenge}

\author{Massimiliano Conte\\
{\tt\small massimiliano.conte.2@studenti.unipd.it}
% For a paper whose authors are all at the same institution,
% omit the following lines up until the closing ``}''.
% Additional authors and addresses can be added with ``\and'',
% just like the second author.
% To save space, use either the email address or home page, not both
\and
Pierpaolo D'Odorico\\
{\tt\small pierpaolo.dodorico@studenti.unipd.it}
}

\maketitle
%\thispagestyle{empty}

%%%%%%%%% ABSTRACT
%\begin{abstract}
%   The ABSTRACT is to be in fully-justified italicized text, at the top
%   of the left-hand column, below the author and affiliation
%   information. Use the word ``Abstract'' as the title, in 12-point
%   Times, boldface type, centered relative to the column, initially
%   capitalized. The abstract is to be in 10-point, single-spaced type.
%   Leave two blank lines after the Abstract, then begin the main text.
%   Abstract should be no longer than 300 words.
%\end{abstract}

%%%%%%%%% BODY TEXT
\section{Introduction}

The faced problem is the speech recognition challenge, where the goal is to build a system that can automatically recognize spoken words. In particular, the spoken words are given as input in form of audio recordings of one second length, while the output of the system is the text of the recognized words. All the recordings are waveform of one of the following 8 words:
\begin{itemize}
	\itemsep-0.3em 
	\item Down;
	\item Go;
	\item Left;
	\item Off;
	\item On;
	\item Right;
	\item Stop;
	\item Up.
\end{itemize}
This task is a key component in many artificial intelligence services, such as virtual assistants, and more generally speech recognition is part of the natural language processing domain.\\
Our work begin after the features extraction provided us by the professors, in form of log mel-spectrogram, that is a bi-dimensional representation of the recordings, involving frequency and time. After that, the images of the spectrograms are first resized and than reshaped in 1024-dimensional vectors.\\
We have faced the problem by applying several machine learning methods, from the easier ones to the more complicated ones, and choosing the best performing method based on accuracy. This led us to a system capable of classifying spoken words with an accuracy of more that 95\%. 

\section{Dataset}

The used dataset is a reduced version of TensorFlow Speech Commands Dataset v0.0.1 \cite{tensorflow2015-whitepaper}. The data provided us is alredy divided in training and validation set. The training set is composed of 1600 samples, 200 for each of the 8 classes, and the validation set is composed of 109 samples.
Every recording has a sampling frequency of 16 kHz. The hop length used for constructing the log Mel-spectrogram is 512, while it is not specified the window size. The feature extraction mimics the humany auditory system, providing a representation of the audio taking into account the fact that humans perceve both the frequencies and the amplitude of the sound logarithmically. The whole preocess of extracting the features for each recording can be summarized in:
\begin{itemize}
	\itemsep0em 
	\item \textbf{Create the windows}, by sampling the recording and making hops of size 512;
	\item \textbf{Compute the discrete Fuorier transform} for each window, using the fast Fourier transform algorithm;
	\item \textbf{Convert to the Mel scale}: changing the representation from of the frequencies from Hz to the Mel scale (the scale of pitches judged by listeners to be equal in distance one from another, a kind of human perceving frequency scale);
	\item \textbf{Create the features}, by computing the log Mel-spectrogram (that is an image), resizing and reshaping it to a 1024 dimensional vector.
\end{itemize}
The dataset provided us alredy contains the extracted features and the correct class of the samples. The dataset doesn't have unbalanced classes problem.

\section{Method}
The methods used in this project are machine learning tecniques that are well suited for multiclass classification tasks. We tried the performance of various methods under different configurations, starting from the basic models and ending with the more complicated ones. Many models we tried are based on binary classification, the way we used such models are using the one-vs-all strategy, i. e. training one binary classifier per class and than predict the instances by loocking at the classifier that maximizes the confidence score.
\subsection{List of used methods}
Every method can be tuned by changing some hyperparameters. In the following table we reported the tecniques, which of those hyperparameters we taked into account for selecting the best configuration, and how that method handle multiclass classification.
The tecniques we tried are:
\begin{table}
	\begin{center}
		\resizebox{\columnwidth}{!}{
		\begin{tabular}{|c|c|c|}
			\hline
			Method & Hyperparameters & handling \\
			\hline\hline
			
			
			
			Linear Classification & C : Regularization & One-vs-all \\
			
			\hline
			Logistic Regression & C : Regularization & One-vs-all \\
			
			\hline
			K-Neighbors Classifier  & K : number of neighbors & Majority vote of  \\
			&& the K-neighbors \\
			
			\hline
			Classification tree & Max depth; & Naturally handle multiclass \\
			&Min samples leaf;&\\
			&Min impurity decrease&\\
			
			\hline
			Random forest & Number of trees & Naturally handle multiclass \\
			
			\hline
			Support Vector Machine & C : Regularization; & One-vs-all \\
			&Type of kernel&\\
			
			\hline
			Neural network &  & Softmax activation \\
			&&on the output layer\\ 
			
			\hline
		\end{tabular}}
	\end{center}
	\caption{Results. Ours is better.}
	\label{mytable}
\end{table}

















\begin{itemize}
	\item Introduction (20\%): describe the problem you are working on, why it's important, what are your goals, and provide also an overview of your main results.
	\item Dataset (20\%): describe the data you are working with for your project. What type of data is it? Where did it come from? How much data are you working with? Did you have to do any preprocessing, filtering, etc., and why?
	\item Method (30\%): discuss your approach for solving the problems that you set up in the introduction. Why is your approach the right thing to do? Did you consider alternative approaches? It may be helpful to include figures, diagrams, or tables to describe your method or compare it with others.
	\item Experiments (30\%): discuss the experiments that you performed. The exact experiments will vary depending on the project, but you might compare with prior work, perform an ablation study to determine the impact of various components of your system, experiment with different hyperparameters or architectural choices. You should include graphs, tables, or other figures to illustrate your experimental results.
\end{itemize}	

%------------------------------------------------------------------------
\section{Formatting your paper}

All text must be in a two-column format. The total allowable width of the
text area is $6\frac78$ inches (17.5 cm) wide by $8\frac78$ inches (22.54
cm) high. Columns are to be $3\frac14$ inches (8.25 cm) wide, with a
$\frac{5}{16}$ inch (0.8 cm) space between them. The main title (on the
first page) should begin 1.0 inch (2.54 cm) from the top edge of the
page. The second and following pages should begin 1.0 inch (2.54 cm) from
the top edge. On all pages, the bottom margin should be 1-1/8 inches (2.86
cm) from the bottom edge of the page for $8.5 \times 11$-inch paper; for A4
paper, approximately 1-5/8 inches (4.13 cm) from the bottom edge of the
page.

%-------------------------------------------------------------------------
\subsection{Margins and page numbering}

All printed material, including text, illustrations, and charts, must be kept
within a print area 6-7/8 inches (17.5 cm) wide by 8-7/8 inches (22.54 cm)
high.
Page numbers should be in footer with page numbers, centered and .75
inches from the bottom of the page and make it start at the correct page
number rather than the 4321 in the example.  To do this fine the line (around
line 23)
\begin{verbatim}
%\ifcvprfinal\pagestyle{empty}\fi
\setcounter{page}{4321}
\end{verbatim}
where the number 4321 is your assigned starting page.

Make sure the first page is numbered by commenting out the first page being
empty on line 46
\begin{verbatim}
%\thispagestyle{empty}
\end{verbatim}


%-------------------------------------------------------------------------
\subsection{Type-style and fonts}

Wherever Times is specified, Times Roman may also be used. If neither is
available on your word processor, please use the font closest in
appearance to Times to which you have access.

MAIN TITLE. Center the title 1-3/8 inches (3.49 cm) from the top edge of
the first page. The title should be in Times 14-point, boldface type.
Capitalize the first letter of nouns, pronouns, verbs, adjectives, and
adverbs; do not capitalize articles, coordinate conjunctions, or
prepositions (unless the title begins with such a word). Leave two blank
lines after the title.

AUTHOR NAME(s) and AFFILIATION(s) are to be centered beneath the title
and printed in Times 12-point, non-boldface type. This information is to
be followed by two blank lines.

The ABSTRACT and MAIN TEXT are to be in a two-column format.

MAIN TEXT. Type main text in 10-point Times, single-spaced. Do NOT use
double-spacing. All paragraphs should be indented 1 pica (approx. 1/6
inch or 0.422 cm). Make sure your text is fully justified---that is,
flush left and flush right. Please do not place any additional blank
lines between paragraphs.

Figure and table captions should be 9-point Roman type as in
Table~\ref{mytable}. Short captions should be centred.

\noindent Callouts should be 9-point Helvetica, non-boldface type.
Initially capitalize only the first word of section titles and first-,
second-, and third-order headings.

FIRST-ORDER HEADINGS. (For example, {\large \bf 1. Introduction})
should be Times 12-point boldface, initially capitalized, flush left,
with one blank line before, and one blank line after.

SECOND-ORDER HEADINGS. (For example, { \bf 1.1. Database elements})
should be Times 11-point boldface, initially capitalized, flush left,
with one blank line before, and one after. If you require a third-order
heading (we discourage it), use 10-point Times, boldface, initially
capitalized, flush left, preceded by one blank line, followed by a period
and your text on the same line.

%-------------------------------------------------------------------------
\subsection{Footnotes}

Please use footnotes\footnote {This is what a footnote looks like.  It
often distracts the reader from the main flow of the argument.} sparingly.
Indeed, try to avoid footnotes altogether and include necessary peripheral
observations in
the text (within parentheses, if you prefer, as in this sentence).  If you
wish to use a footnote, place it at the bottom of the column on the page on
which it is referenced. Use Times 8-point type, single-spaced.


%-------------------------------------------------------------------------
\subsection{References}

List and number all bibliographical references in 9-point Times,
single-spaced, at the end of your paper. When referenced in the text,
enclose the citation number in square brackets, for
example~\cite{Authors14}.  Where appropriate, include the name(s) of
editors of referenced books.

\begin{table}
\begin{center}
\begin{tabular}{|l|c|}
\hline
Method & Frobnability \\
\hline\hline
Theirs & Frumpy \\
Yours & Frobbly \\
Ours & Makes one's heart Frob\\
\hline
\end{tabular}
\end{center}
\caption{Results. Ours is better.}
\label{mytable}
\end{table}

%-------------------------------------------------------------------------
\subsection{Illustrations, graphs, and photographs}

All graphics should be centered.  Please ensure that any point you wish to
make is resolvable in a printed copy of the paper.  Resize fonts in figures
to match the font in the body text, and choose line widths which render
effectively in print.  Many readers (and reviewers), even of an electronic
copy, will choose to print your paper in order to read it.  You cannot
insist that they do otherwise, and therefore must not assume that they can
zoom in to see tiny details on a graphic.

When placing figures in \LaTeX, it's almost always best to use
\verb+\includegraphics+, and to specify the  figure width as a multiple of
the line width as in the example below
{\small\begin{verbatim}
   \usepackage[dvips]{graphicx} ...
   \includegraphics[width=0.8\linewidth]
                   {myfile.eps}
\end{verbatim}
}

%{\small
%\bibliographystyle{ieee_fullname}
%\bibliography{egbib}
%}

\printbibliography

\end{document}
